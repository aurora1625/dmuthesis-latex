
\chapter{模板的安装使用}\label{ch:install}

\section{安装 \TeX 系统}
\subsection{Windows}
考虑到Windows系统用户大多数使用CTex(基于MiKTeX),这里只说明CTex的安装方法。
\begin{enumerate}
\item 下载CTex v2.9完整版,\url{http://www.ctex.org/CTeXDownload};
\item 按照第\ref{sect:compile}节的方法编译你的论文即可。
\end{enumerate}

\subsection{Mac OSX}

完整安装MacTeX套装即可。

\subsection{Linux}

完整安装TeXLive最新版即可。

\subsection{有关字体的说明}
Windows用户请略过这小节的内容。

 由于此模板的目的是尽量与研究生院发布的Word模板接近,所以请MAC和Linux系统用户
 安装Windows系统下的宋体,黑体,楷体和仿宋。

除此之外,也可以自行安装Adobe系列字体,或者其他和Windows下宋体、黑体等类似的
字体,然后在\texttt{dmuthesis.cls}中 \texttt{setCJKmainfont} 
等命令中替换成相应字体。


\section{编译你的论文}\label{sect:compile}

论文的中英文封面和原创性声明需要使用Word编辑,并且在每页之后添加一个空白页(Ctrl+Enter),然后将前六页保存成pdf文件——cover.pdf。

之后在body文件夹用文本编辑器打开各个源文件进行编辑。

\subsection{文本编辑器}

不要使用CTex套装中预制了WinEdt编辑器,否则后果自负。
由于WinEdt使用GBK编码。插入参考文献时需要配合另外的软件
使用,大多数软件默认的编码都是UTF8,所以不推荐WinEdt,我们推荐使用
TexStudio、Vim+LaTexSuite等支持UTF8编码的专用LaTeX编辑器进行编辑。
也可以使用Sublime Text 2、Notepad++等其他纯文本编辑器。

文本编辑器推荐列表(依照推荐顺序排序):

\begin{itemize}
\item IDE组
\begin{enumerate}
	\item TeXstudio
	\item Texmaker
	\item TeXnicCenter
\end{enumerate}
\item{纯文本编辑器组}
\begin{enumerate}
\item Vim+LaTeXSuite (神器)
\item Sublime Text 2
\end{enumerate}
\end{itemize}


\subsection{快捷编译}

如果确保使用的\LaTeX 命令没有错误,可以使用模板附带的批处理命令进行编译。
\subsubsection{Windows下编译}

双击 \texttt{run.bat}文件进行编辑。

\subsubsection{Mac OSX和Linux下编译}

在你最喜欢的bash下面运行\texttt{make} 编译。
或者运行\texttt{sh run.sh}编译。

\subsection{使用命令行编译}

这种方式在所有系统下都可以进行。打开命令行工具,进入模板所在文件夹,按照下面顺序依次输入:
\begin{verbatim}
xelatex dmuthesis
bibtex dmuthesis
xelatex dmuthesis
xelatex dmuthesis
\end{verbatim}
为了生成参考文献和书签以及交叉引用,所有的\LaTeX 文件都需要编译4次,\XeLaTeX 
文件也不例外。


